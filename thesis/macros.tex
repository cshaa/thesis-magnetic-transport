%%% This file contains definitions of various useful macros and environments %%%
%%% Minor tweaks of style

% These macros employ a little dirty trick to convince LaTeX to typeset
% chapter headings sanely, without lots of empty space above them.
% Feel free to ignore.
\makeatletter
\def\@makechapterhead#1{
  {\parindent \z@ \raggedright \normalfont
   \Huge\bfseries \thechapter. #1
   \par\nobreak
   \vskip 20\p@
}}
\def\@makeschapterhead#1{
  {\parindent \z@ \raggedright \normalfont
   \Huge\bfseries #1
   \par\nobreak
   \vskip 20\p@
}}
\makeatother

% This macro defines a chapter, which is not numbered, but is included
% in the table of contents.
\def\chapwithtoc#1{
\chapter*{#1}
\addcontentsline{toc}{chapter}{#1}
}

% Draw black "slugs" whenever a line overflows, so that we can spot it easily.
% \overfullrule=1mm

%%% Macros for definitions, theorems, claims, examples, ... (requires amsthm package)

\theoremstyle{plain}
\newtheorem{thm}{Theorem}
\newtheorem{lemma}[thm]{Lemma}
\newtheorem{claim}[thm]{Claim}
\newtheorem{defn}[thm]{Definition}

\theoremstyle{remark}
\newtheorem*{cor}{Corollary}
\newtheorem*{rem}{Remark}
\newtheorem*{example}{Example}

%%% An environment for proofs

\newenvironment{myproof}{
  \par\medskip\noindent
  \textit{Proof}.
}{
\newline
\rightline{$\qedsymbol$}
}

%%% An environment for typesetting of program code and input/output
%%% of programs. (Requires the fancyvrb package -- fancy verbatim.)

\DefineVerbatimEnvironment{code}{Verbatim}{fontsize=\small, frame=single}

%%% The field of all real and natural numbers
\renewcommand{\C}{\mathbb{C}}
\newcommand{\K}{\mathbb{K}}
\newcommand{\R}{\mathbb{R}}
\newcommand{\Z}{\mathbb{Z}}
\newcommand{\N}{\mathbb{N}}

%%% Useful operators for statistics and probability
\DeclareMathOperator{\pr}{\textsf{P}}
\DeclareMathOperator{\E}{\textsf{E}\,}
\DeclareMathOperator{\var}{\textrm{var}}
\DeclareMathOperator{\sd}{\textrm{sd}}

%%% Transposition of a vector/matrix
\newcommand{\T}[1]{#1^\top}

%%% Various math goodies
\newcommand{\goto}{\rightarrow}
\newcommand{\gotop}{\stackrel{P}{\longrightarrow}}
\newcommand{\maon}[1]{o(n^{#1})}
\newcommand{\abs}[1]{\left|{#1}\right|}
\newcommand{\isqr}[1]{\frac{1}{\sqrt{#1}}}

%%% Various table goodies
\newcommand{\pulrad}[1]{\raisebox{1.5ex}[0pt]{#1}}
\newcommand{\mc}[1]{\multicolumn{1}{c}{#1}}


%%% Custom commands
\renewcommand{\d}[1]{\, \mathrm{d}#1 \,}
\renewcommand{\i}{\mathrm{i}}
\newcommand{\Cont}{C}
\newcommand{\Hilb}{\mathcal{H}}
\newcommand{\Schwa}{\mathcal{S}}
\newcommand{\Domain}{\mathrm{D}}
\newcommand{\Fourier}{\mathcal{F}}
\newcommand{\Nbhood}{\mathcal{U}}
\newcommand{\norm}[1]{\big\lVert #1 \big\rVert}
\newcommand{\Hf}{\mathscr{H}}
\newcommand{\Af}{\mathscr{A}}
\newcommand{\ve}{\varepsilon}
\newcommand{\hypF}{\vphantom{\mathrm{F}}_1\hspace{-0.7pt}\mathrm{F}_{\!1}}
\newcommand{\supp}{\operatorname{supp}}

\newcommand{\Sp}{\sigma}
\newcommand{\SpP}{\sigma_{\mathrm{p}}}
\newcommand{\SpAc}{\sigma_{\mathrm{ac}}}
\newcommand{\SpSc}{\sigma_{\mathrm{sc}}}
\newcommand{\SpD}{\sigma_{\mathrm{disc}}}
\newcommand{\SpE}{\sigma_{\mathrm{ess}}}

\newcommand{\tildeA}{\overset{\;\,\scriptscriptstyle\sim}{A}}
\newcommand{\tildeH}{\overset{\;\,\scriptscriptstyle\sim}{H}}

\newcommand{\dd}[2]{\frac{\mathrm{d} #1}{\mathrm{d} #2}}
\newcommand{\pd}[2]{\frac{\partial  #1}{\partial  #2}}
\newcommand{\e}[1]{\mathrm{e}^{#1}}

\newcommand{\coloneqq}{:=}
\newcommand{\eqqcolon}{=:}

% Manual numbering of equations
\newcommand\numberthis{\addtocounter{equation}{1}\tag{\theequation}}


% Whitespace magic
\def\ph{\phantom}
\def\vph{\vphantom}
\def\hph{\hphantom}
\def\rzw{\mathrlap}
\def\lzw{\mathllap}
\def\czw{\mathclap}
\newcommand*{\mask}[2]{\mathord{\makebox[\widthof{\(#1\)}]{\(#2\)}}}
\newcommand{\scalemath}[2]{\scalebox{#1}{\mbox{\ensuremath{\displaystyle #2}}}}

