\section{Minimal energies for the Robin boundary condition}
\label{apdx-pythom}
In this section we list the numerically computed minimal energies for the Landau Hamiltonian in a half-plane with a Robin boundary.

\ph{.}
\begin{minipage}{\linewidth-25pt}
    \vspace{\baselineskip}
    \centering
    \begin{tabular}{ c|c }
        \bfseries $\alpha / \sqrt{b}$ &
        \bfseries $\epsilon / b \pm 0.001$
		\csvreader[ head to column names ]{../num/robin-lowest-energy-table.csv}{}
        {
            \csviffirstrow{\\\hline}{\\}
            $\boundary$ & $\energy$
        }
    \end{tabular}
    \vspace{0.2\baselineskip}
\end{minipage}

\noindent
The following naïve algorithm was used to compute the energies. Better performance could be achieved in many ways, for example by using the Newton's method to find the solutions of $F(...) = 0$ and the Levenberg-Marquardt algorithm to find the minimum.

\lstinputlisting[language=Python]{../num/robin-lowest-energy.py}
\begin{lstlisting}[language=Python]

# set up limits and steps
step_e = 0.1
step_p = 0.01
step_a = 0.1
min_a = -2
max_a = 2

estimates = [
    (a, e := find_minimum_energy(a, e, step_p, step_e))
    for a in float_range(min_a, max_a, step_a)
]

print_estimates(estimates)
\end{lstlisting}
