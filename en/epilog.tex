\chapter*{Conclusion}
\addcontentsline{toc}{chapter}{Conclusion}

The aim of this work was to summarize the current knowledge of translationally invariant magnetic Hamiltonians and to complement that knowledge with information about two previously unstudied systems.

One of the systems that we studied was the Landau Hamiltonian with a Dirac delta-interaction of strength $\alpha$ supported on a line. We verified that the Hamiltonian is self-adjoint and bounded from below, and that it is a fibred operator and the spectrum of its fibre is discrete and nondegenerate. Then we found implicit expressions for the eigenvalues and eigenfunctions of the fibre. Finally, we proved that: for a repulsive interaction ($\alpha>0$), there are spectral bands that start on each Landau level and extend up, for an attractive interaction ($\alpha < 0$) the spectral bands start on each Landau level and extend down. In both cases the spectral bands never merge, there is a forbidden energy gap between each two neighbouring Landau levels. The spectrum of the Hamiltonian is absolute continuous; its singular continuous and pure point spectra are empty.

The other system that we examined was the Landau Hamiltonian on a half-plane with a Robin boundary condition. As with the delta-interaction, we verified the self-adjointness and boundedness from below, and that the fibre Hamiltonian has a discrete nondegenerate spectrum. Then we used the properties of the fibre to prove that: the spectrum of the Hamiltonian is absolute continuous, starting from a minimum below the first Landau level and extending up ad infinitum. We have numerically calculated estimates of the minimum energy for different values of the boundary parameter $\alpha$.

Topics for further studies of the spectral properties of translationally invariant magnetic Hamiltonians are far from exceeded. Regarding the two Hamiltonians we studied, future work might investigate the stability of their spectra under perturbations by various classes of disorder potentials, or find upper and lower limits for the minimum energy of the Robin half-plane Hamiltonian. Other Landau Hamiltonians with potential perturbations which, to the author's knowledge, are yet to be studied include: multiple $\delta$-interactions on parallel lines, a $\delta'$-interaction\footnote{The plausibility of a $\delta'$-interaction in one dimension is discussed in Chapter III.3 of \cite{Albeverio2005}.} supported on a line or on multiple parallel lines, and a strip with Neumann or Robin boundaries.
