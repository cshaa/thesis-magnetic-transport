\chapter*{List of symbols}
\addcontentsline{toc}{chapter}{List of symbols}

{\renewcommand{\arraystretch}{1.3}
\begin{table}[h!]
    \begin{tabular}{c|p{0.7\linewidth}}
        $C^k(\Omega, \K)$ & The space of functions $\Omega \subseteq \R \to \K$ with $k$ continuous derivatives. \\
        $C^\infty_0(\Omega, \K)$ & The space of $C^\infty$ functions with compact support in $\Omega$. \\
        $\Domain(T)$ & The domain of operator $T$, usually dense in $\Hilb$. \\
        $D_\nu(w)$ & The parabolic cylinder function. \\
        $\Fourier$ & The Fourier-Plancherel operator on $L^2(\R)$. \\
        $\hypF(\alpha, \beta; z)$ & The confluent hypergeometric function of the first kind. \\
        $\Hilb$ & A separable Hilbert space. \\
        $H, \Hf(\xi)$ & A Hamiltonian operator; a fiber of the Hamiltonian. \\
        $H_n(x)$ & The $n$-th Hermite polynomial. \\
        $L^p(M, \d{\mu}, V)$ & The space of $p$-integrable functions from measure space $(M, \mu)$ to vector space $V$. Specifically for $p=2$, a Hilbert space with inner product $(\psi, \phi)_{L^2} = \int_M (\psi, \phi)_V \d{\mu}$. \\
        $L^p(\Omega)$ & As above, but $M=\Omega \subseteq \R^N$, $\mu$ is the Lebesgue measure and $V=\C$. \\
        $L^1_{\mathrm{loc}}(\Omega)$ & The space of functions that are $L^1(K)$ for every compact $K \subset \Omega$. \\
        $\N, \; \N_0$ & The set of positive integers; the set of non-negative integers. \\
        $\vv P, P_x, P_y, P_z$ & Momentum operator – a self-adjoint operator, such that $P_x \, f(x,...) = -\i \pd{}{x} f(x,...) \: .$ \\
        $\vv Q, Q_x, Q_y, Q_z$ & Position operator – a self-adjoint operator, such that $Q_x \, f(x,...) = x \, f(x,...) \: .$ \\
        $W^{k,p}(\Omega)$ & The Sobolev space – the space of integrable functions $f$, such that $f^{(\alpha)} \in L^p(\Omega)$, where $\alpha$ is a multi-index and $|\alpha| \leq k$. \\
        $\Gamma(z)$ & The gamma function. \\
        $\mu$ & A $\sigma$-finite measure, usually the Lebesgue measure. \\
        \makecell[tc]{$\Sp(T), \SpP(T),$\\$\SpAc(T), \SpSc(T)$} & The spectrum of normal operator $T$; the point, absolutely continuous, singular continuous spectrum of $T$ \\
        $\nabla, \; \nabla \times, \; \Delta$ & Gradient, rotation, Laplace operator. \\
        $\Delta^\Omega_{\mathrm{D}}, \Delta^\Omega_{\mathrm{D},A}$ & The Dirichlet Laplacian, defined on functions from $L^2(\Omega)$ with a Dirichlet boundary condition; a “magnetic” Dirichlet Laplacian given by the vector potential $A$.
    \end{tabular}
\end{table}
}

\chapter*{Introduction}
\addcontentsline{toc}{chapter}{Introduction}

