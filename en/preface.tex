\chapter*{List of symbols}
\addcontentsline{toc}{chapter}{List of symbols}
\noindent\vspace{-2\baselineskip}
{\renewcommand{\arraystretch}{1.3}
\begin{table}[h!]
    \begin{tabular}{c|p{0.7\linewidth}}
        $A |_\Omega$ & Restriction of an operator $A$ to a subspace $\Omega \subset \Domain(A)$. \\
        \makecell[tc]{$\Cont^k(\Omega, \K)$, \\ $\Cont(\Omega, \K)$} & The space of functions $\Omega \subseteq \R \to \K$ with $k$ continuous derivatives. The space of continuous functions. \\
        $\Cont^\infty_0(\Omega, \K)$ & The space of $C^\infty$ functions with compact support in $\Omega$. \\
        $\Domain(A)$ & The domain of operator $A$, usually dense in $\Hilb$. \\
        $D_\nu(w)$ & The parabolic cylinder function. \\
        $\Fourier$ & The Fourier-Plancherel operator on $L^2(\R)$. \\
        $\hypF(\alpha, \beta; z)$ & The confluent hypergeometric function of the first kind. \\
        $\Hilb$ & A separable Hilbert space. \\
        $H, \; \Hf(\xi)$ & A Hamiltonian operator; a fibre of the Hamiltonian. \\
        $H_n(x)$ & The $n$-th (physicist's) Hermite polynomial. \\
        $L^p(M, \d{\mu}, V)$ & The space of $p$-integrable functions from measure space $(M, \mu)$ to vector space $V$. Specifically for $p=2$, a Hilbert space with inner product $(\psi, \phi)_{L^2} = \int_M (\psi, \phi)_V \d{\mu}$. \\
        $L^p(\Omega)$ & As above, but $M=\Omega \subseteq \R^N$, $\mu$ is the Lebesgue measure and $V=\C$. \\
        $L^p_\rho(\Omega)$ & A weighted Lebesgue space, a shorthand for $L^p(\Omega \subseteq \R^N, \d{\mu}, \C)$, where $\rho$ is a real function and $\mu(M) = \int_M \rho \d{\lambda}$ is a rescaling of the Lebesgue measure $\lambda$. \\
        $L^1_{\mathrm{loc}}(\Omega)$ & The space of functions that are $L^1(K)$ for every compact $K \subset \Omega$. \\
        $\N, \; \N_0$ & The set of positive integers; the set of non-negative integers. \\
        $\vv P, P_x, P_y, P_z$ & Momentum operator – a self-adjoint operator, such that $P_x \, f(x,...) = -\i \pd{}{x} f(x,...) \: .$ \\
        $\vv Q, Q_x, Q_y, Q_z$ & Position operator – a self-adjoint operator, such that $Q_x \, f(x,...) = x \, f(x,...) \: .$ \\
        $W^{k,p}(\Omega)$ & The Sobolev space – the space of integrable functions $f$, such that $f^{(\alpha)} \in L^p(\Omega)$, where $\alpha$ is a multi-index and $|\alpha| \leq k$. \\
        $\Gamma(z)$ & The gamma function. \\
        $\mu$ & A $\sigma$-finite measure, usually the Lebesgue measure. \\
        \makecell[tc]{$\Sp(T), \SpP(T),$\\$\SpAc(T), \SpSc(T),$\\$\SpD(T), \SpE(T)$} & The spectrum of normal operator $T$; the point, absolutely continuous, singular continuous, discrete, essential spectrum of $T$. $\Sp(T) = \SpP \cup \SpAc \cup \SpSc = \SpD \cup \SpE \:$. \\
        $\nabla, \; \nabla \times, \; \Delta$ & Gradient, rotation, Laplace operator. \\
        $\Delta^\Omega_{\mathrm{D}}, \Delta^\Omega_{\mathrm{D},A}$ & The Dirichlet Laplacian, defined on functions from $L^2(\Omega)$ with a Dirichlet boundary condition; a “magnetic” Dirichlet Laplacian given by the vector potential $A$.
    \end{tabular}
\end{table}
}

\chapter*{Introduction}
\addcontentsline{toc}{chapter}{Introduction}

In this thesis we will investigate the spectral properties of Schrödinger operators of the form $(-\i\nabla + \vec{A}(x))^2 + V(x)$ on $L^2(\Omega)$ where $\Omega$ is either (a subset of) $\R^2$, or a thin layer in $\R^3$. These operators are of great interest, as they represent the Hamiltonians in single-particle models of two-dimensional magnetic systems.

Classically, a magnetic field is known to have a localizing effect on charged particles – in a homogeneous field, for example, particles get stuck in circular trajectories due to the Lorentz force. Nonetheless, there are many ways in which such localization can be overcome: if we introduce a potential wall, the particle will bounce off of it, propagating along the wall in one direction in a phenomenon called \textit{skipping}; if we introduce a step in the magnetic field, the radius of circular motion will be different in different parts of the plane leading to so-called \textit{snake orbits} which also transport the particle along the step.

All three of these examples translate well into quantum mechanics. In the case of a homogeneous magnetic field, we get the Landau Hamiltonian with a complete set of (localized) eigenstates. A wide range of wall potentials has been shown to lead to transport by \cite{Macris1999} and \cite{Frohlich2000}. The quantum-classical correspondence of snake orbits was investigated by \cite{Reijniers2000}. However, there are numerous other examples of magnetic transport, and not all of them have a clear classical analogy. The telltale sign of Hamiltonians which allow for magnetic transport is that they have a non-empty absolute continuous spectrum – that is why we are interested in the spectral properties of such operators.

We will limit ourselves to systems where both the scalar potential and the magnetic field are translationally invariant, as they are the obvious choice for studying extended motion in one direction. A more general approach would also allow for a bounded \textit{disorder potential}, which is not translationally invariant. While this is not our focus, some of the results we include in chapter~\ref{chapter-known-results} account for such potentials.

The work is organized as follows: In chapter~\ref{chapter-formulation} we give some more motivation and define the classes of Hamiltonians we will investigate. Then we introduce the concept of the direct integral, which is an invaluable tool when working with translationally invariant operators on $L^2$. Finally, we list all the standard theorems of operator theory that we will use further on. In chapter~\ref{chapter-known-results} we include an extensive list of theorems that have already been proven about the spectral properties of translationally invariant magnetic Hamiltonians. The chapter is divided into three sections according to whether the systems include 1) a nonzero scalar potential, 2) a nonhomogeneous magnetic field or 3) a geometric deformation of the two-dimensional layer. Each section is sorted by the date of publication of the results. In chapter \ref{chapter-dirac} we investigate the Landau Hamiltonian with a Dirac delta-interaction supported on a line, in chapter \ref{chapter-robin} we examine the Landau Hamiltonian in a half-plane with Robin boundary condition. To the author's knowledge, the spectra of these two Hamiltonian have not yet been rigorously studied. In both chapters, we prove that the Hamiltonian in question is self-adjoint and bounded from below, then we proceed to study its spectral properties.
